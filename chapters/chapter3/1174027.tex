\section{1174027 - Harun Ar - Rasyid}
\subsection{Teori}
\begin{enumerate}
	\item Jelaskan apa itu random forest, sertakan gambar ilustrasi buatan sendiri.
	\hfill\break
	Random Forest merupakan algoritma yang digunakan terhadapap klasifikasi data dalam jumlah yang besar. Klasifikasi pada random forest dilakukan dengan penggabungan dicision tree dengan melakuakn training terhadap sempel data yang dimiliki. Semakin banyak dicision tree maka data yang di dapat akan semakin akurat. Untuk gambar Random Forest dapat dilihat dibawah ini :
	\begin{figure}[H]
	\centering
		\includegraphics[width=4cm]{figures/1174027/3/1.png}
		\caption{Random Forest.}
	\end{figure}
	\item Jelaskan cara membaca dataset kasus dan artikan makna setiap file dan isi field masing-masing file.
	\hfill\break
	\begin{itemize}
		\item Pertama download dataset terlebih dahulu lalu buka dengan menggunakan software spyder guna melihat isi dari dataset tersebut.
		\item Data tersebut memiliki extensi file bernama .txt dan didalamnya terdapat class dari field.
		\item Misalnya saja pada data jenis burung memiliki file index dan angka, dimana index berisi angka yang memiliki makna berupa jenis burung.
		\item bahkan nama burung sedangkan field memiliki isi nilai berupa 0 dan 1 yang dimana sifatnya boolean atau Ya dan Tidak.
		\item Hal ini dikarenakan komputer hanya dapat membaca bilangan biner maka dari itu field yang di isikan berupa angka.
		\item Artinya angka 0 berarti tidak dan angka 1 berarti Ya.
	\end{itemize}
	\item Jelaskan apa itu cross validation.
	\hfill\break
	Cross Validation merupakan sebuah teknik validasi model yang berfungsi untuk melakukan penilaian bagaimana hasil analisis statistik akan digeneralisasi ke data yang lebih independen. Cross validation digunakan dengan tujuan prediksi, dan bila kita ingin memperkirakan seberapa akurat model model prediksi yang dilakukan dalam sebuah praktek. Tujuan dari cross validation yaitu untuk mendefinisikan dataset guna menguju dalam fase pelatihan untuk membatasi masalah seperti overfitting dan underfitting serta mendapatkan wawasan tentang bagaimana model akan digeneralisasikan ke set data independen.
	\item Jelaskan apa arti score 44 \% pada random forest, 27\% pada decission tree dan 29 \% dari SVM.
	\hfill\break
	Dimana Score 44 \% diperoleh dari hasil pengelohan dataset jenis burung. Dimana akan dilakukan proses pembagian data testing dan data training lalu diproses dan menghasilkan score sebanyak 44 \% dimana menjelaskan bahwa score tersebut digunakan sebagai pembanding dalam tingkat keakuratannya. Pada dicision tree akan memperoleh data lebih kecil yaitu sebanyak 27 \% hal ini dikarenakan data yang diolah menggunakan dicision tree dibagi menjadi beberapa tree dan lalu disimpulkan untuk mendapatkan data yang akurat. Pada SVM akan memperoleh score sebanyak 29 \% hal ini dikarenakan data yang dimiliki masih bernilai netral sehingga tingkat keakuratannya masih belum jelas.
	\item Jelaskan bagaimana cara membaca confusion matriks dan contohnya memakai gambar atau ilustrasi sendiri.
	\hfill\break
	Untuk membaca confusion matriks dapat menggunakan source code sebagai berikut :
	\lstinputlisting[firstline=233, lastline=237]{src/1174027/3/1174027_teori.py}
	\hfill\break
	Dimana numpy akan mengurus semua data yang berhubungan dengan matrix. Pada source code tersebut digunakan dalam melakukan read pada dataset burung dengan menggunakan metode confusion matrix. Dalam confusion matrix memiliki 4 istilah yaitu True Positive yang merupakan data posotif yang terditeksi benar, True Negatif yang merupakan data negatif akan tetapi terditeksi benar, False Positif merupakan data negatif namun terditeksi sebagai data positif, False Negatif merupakan data posotif namun terditeksi sebagai data negatif. Adapun contoh hasil read dataset menggunakan confusion matrix dapat dilihat dibawah ini :
	\begin{figure}[H]
	\centering
		\includegraphics[width=2cm]{figures/1174027/3/2.png}
		\caption{Confusion Matriks.}
	\end{figure}
	\item Jelaskan apa itu voting pada random forest disertai dengan ilustrasi gambar sendiri
	\hfill\break
	Voting pada random forest ialah sebuah proses pemilihan dari tree yang dimana akan dimunculkan hasilnya dan disimpulkan menjadi informasi yang pasti. Untuk lebih jelasnya saya akan memberikan sebuah contoh bagaimana voting bekerja : 
	\begin{figure}[H]
	\centering
		\includegraphics[width=4cm]{figures/1174027/3/3.png}
		\caption{Decision Tree.}
	\end{figure}
	Dimana ditunjukkan pada gambar diatas terdapat 3 tree. Dalam tree tersebut akan dilakukan proses voting. Saya akan memberikan contoh kasus, dimana akan diadakan voting untuk menentukan sebuah laptop. Dalam tree akan diberikan sejumlah data misalnya saja data tersebut berupa gambar, yang dimana data tersebut akan dipilih dengan cara voting. Hasil voting akhir dari setiap tree menunjukkan laptop asus, yang berarti kesimpulan dari data yang telah diberikan menyatakan gambar tersebut adalah laptop asus. Bagaimana apabila terjadi perbedaan data misalnya saja pada tree 1 dan 2 menunyatakan laptop asus sedangkan pada tree 3 menyatakan laptop HP, maka kesimpulan yang di ambil adalah laptop asus dikarenakan hasil voting terbanyak adalah laptop asus.
\end{enumerate}
\subsection{Praktek}
\begin{enumerate}
	\item Buat aplikasi sederhana menggunakan pandas dan jelaskan arti setiap baris kode yang dibuat(harus beda dengan teman sekelas).
	\hfill\break
	\lstinputlisting[firstline=9, lastline=12]{src/1174027/3/1174027_praktek.py}
	Kode di atas digunakan untuk mengimpor atau mengirim library pandas sebagai pd, Hasilnya adalah sebagai berikut :
	\begin{figure}[H]
	\centering
		\includegraphics[width=4cm]{figures/1174027/3/hasil1.png}
		\caption{Hasil Soal 1.}
	\end{figure}
	\item buat aplikasi sederhana menggunakan numpy dan jelaskan arti dari setiap baris kode yang dibuat(harus beda dengan teman sekelas)
	\hfill\break
	\lstinputlisting[firstline=14, lastline=16]{src/1174027/3/1174027_praktek.py}
	Fungsi sum disini berguna untuk menjumlahkan data. Hasilnya adalah sebagai berikut :
	\begin{figure}[H]
	\centering
		\includegraphics[width=4cm]{figures/1174027/3/hasil2.png}
		\caption{Hasil Soal 2.}
	\end{figure}
	\item buat aplikasi sederhana menggunakan matplotlib dan jelaskan arti dari setiap baris kode(harus beda dengan teman sekelas)
	\hfill\break
	\lstinputlisting[firstline=18, lastline=23]{src/1174027/3/1174027_praktek.py}
	Jalankan source code diatas dengan spyder atau anaconda sehingga hasilnya akan seperti berikut : 
	\begin{figure}[H]
	\centering
		\includegraphics[width=4cm]{figures/1174027/3/hasil3.png}
		\caption{Hasil Soal 3.}
	\end{figure}
	\item jalankan program klasifikasi Random Forest pada bagian teori bab ini. Tunjukkan keluarannya dari komputer sendiri dan artikan maksud setiap luaran yang didapatkan.
	\hfill\break
	\begin{itemize}
		\item Source Code pertama pada random forest berfungsi untuk membaca dataset yang memiliki format text file dengan mendefinisikan variabel yang bernama imgatt. Variabel tersebut berisi value untuk membaca data. 
		\lstinputlisting[firstline=10, lastline=14]{src/1174027/3/1174027_teori.py}
		\item Pada source code berikutnya akan mengembalikan baris teratas dari DataFrame variabel imgatt.
		\lstinputlisting[firstline=19, lastline=19]{src/1174027/3/1174027_teori.py}
		\item Pada output berikutnya akan menampilkan jumlah kolom dan baris dari DataFrame imgatt. 
		\lstinputlisting[firstline=24, lastline=24]{src/1174027/3/1174027_teori.py}
		\item Variabel imgatt2 telah menggunakan function yang bernama pivot guna mengubah kolom jadi baris dan sebaliknya dari DataFrame sebelumnya.
		\lstinputlisting[firstline=29, lastline=29]{src/1174027/3/1174027_teori.py}
		\item Variabel imgatt2 head berfungsi untuk mengembalikan value teratas pada DataFrame imgatt2.
		\lstinputlisting[firstline=34, lastline=34]{src/1174027/3/1174027_teori.py}
		\item Menghasilkan jumlah kolom dan baris pada DataFrame imgatt2.
		\lstinputlisting[firstline=39, lastline=39]{src/1174027/3/1174027_teori.py}
		\item Menunjukkan dalam melakukan pivot yang mana imgid menjadi sebuah index yang unik.
		\lstinputlisting[firstline=44, lastline=47]{src/1174027/3/1174027_teori.py}
		\item Akan melakukan load jawabannya yang berisi apakah burung tersebut termasuk spesies yang mana. Kolom tersebut yaitu imgid dan label.
		\lstinputlisting[firstline=52, lastline=52]{src/1174027/3/1174027_teori.py}
		\item Menunjukkan bahwa jumlah baris sebanyak 11788 dan kolom 1 yang dimana kolom tersebut adalah jenis spesies pada burung.
		\lstinputlisting[firstline=57, lastline=57]{src/1174027/3/1174027_teori.py}
		\item Melakukan join antara imgatt2 dengan imglabels dikarenakan memiliki isi yang sama sehingga akan mendapatkan sebuah data ciri-ciri dan data jawaban sehingga bisa dikategorikan sebagai supervised learning.
		\lstinputlisting[firstline=62, lastline=63]{src/1174027/3/1174027_teori.py}
		\item Melakukan drop pada label yang ada didepan dan akan menggunakan label yang baru di joinkan.
		\lstinputlisting[firstline=68, lastline=69]{src/1174027/3/1174027_teori.py}
		\item Mengecek isi 5 data teratas pada df att.
		\lstinputlisting[firstline=74, lastline=74]{src/1174027/3/1174027_teori.py}
		\item Mengecek isi data teratas dari df label.
		\lstinputlisting[firstline=79, lastline=79]{src/1174027/3/1174027_teori.py}
		\item Membagi 8000 row pertama menjadi data training dan sisanya adalah data testing.
		\lstinputlisting[firstline=84, lastline=90]{src/1174027/3/1174027_teori.py}
		\item Pemanggilan class RandomForestClassifier. Dimana artinya menunjukkan banyak kolom pada setiap tree adalah 50.
		\lstinputlisting[firstline=95, lastline=96]{src/1174027/3/1174027_teori.py}
		\item Menunjukkan hasil prediksi dari Random Forest.
		\lstinputlisting[firstline=101, lastline=101]{src/1174027/3/1174027_teori.py}
		\item Menampilkan besaran akurasi dari prediksi pada Random Forest yang merupakan score perolehan klarifikasi.
		\lstinputlisting[firstline=106, lastline=106]{src/1174027/3/1174027_teori.py}
	\end{itemize}
	\item Jalankan program confusion matrix pada bagian teori bab ini. Tunjukkan keluarannya dari komputer sendiri dan artikan maksud setiap luaran yang didapatkan.
	\hfill\break
	\begin{itemize}
		\item Melakukan import confusion matrix pada library sklearn matriks.
		\lstinputlisting[firstline=116, lastline=118]{src/1174027/3/1174027_teori.py}
		\item Menampilkan isi cm dalam bentuk matrix yang berupa array.
		\lstinputlisting[firstline=123, lastline=123]{src/1174027/3/1174027_teori.py}
		\item Membuat dan menampilkan hasil plot.
		\lstinputlisting[firstline=128, lastline=171]{src/1174027/3/1174027_teori.py}
	\end{itemize}
	\item Jalankan program klasifikasi SVM dan Decission Tree pada bagian teori bab ini. Tunjukkan keluarannya dari komputer sendiri dan artikan maksud setiap luaran yang didapatkan.
	\hfill\break
	\begin{itemize}
		\item Menunjukkan klarifikasi dengan decission tree menggunakan dataset yang sama dan akan memunculkan akurasi prediksi.
		\lstinputlisting[firstline=177, lastline=180]{src/1174027/3/1174027_teori.py}
		\item Menunjukkan klarifikasi dengan SVM menggunakan dataset yang sama dan akan memunculkan akurasi prediksi.
		\lstinputlisting[firstline=185, lastline=188]{src/1174027/3/1174027_teori.py}
	\end{itemize}
	\item Jalankan program cross validaiton pada bagian teori bab ini. Tunjukkan keluarannya dari komputer sendiri dan artikan maksud setiap luaran yang didapatkan.
	\hfill\break
	\begin{itemize}
		\item Menunjukkan hasil cross validation pada Random Forest.
		\lstinputlisting[firstline=193, lastline=195]{src/1174027/3/1174027_teori.py}
		\item Menunjukkan hasil cross validation pada Desiccion Tree.
		\lstinputlisting[firstline=200, lastline=201]{src/1174027/3/1174027_teori.py}
		\item Menunjukkan hasil cross validation pada SVM.
		\lstinputlisting[firstline=206, lastline=207]{src/1174027/3/1174027_teori.py}
	\end{itemize}
	\item Jalankan program pengamatan komponen informasi pada bagian teori bab ini. Tunjukkan keluarannya dari komputer sendiri dan artikan maksud setiap luaran yang didapatkan.
	\hfill\break
	\begin{itemize}
		\item Menunjukkan informasi-informasi tree yang dibuat.
		\lstinputlisting[firstline=212, lastline=225]{src/1174027/3/1174027_teori.py}
		\item Menunjukkan hasil dari plotting komponen informasi sehingga dapat kita baca sebagai grafik 3D.
		\lstinputlisting[firstline=230, lastline=244]{src/1174027/3/1174027_teori.py}
	\end{itemize}
\end{enumerate}
\subsection{Penanganan Error}
\begin{enumerate}
	\item ScreenShoot Error
	\begin{figure}[H]
		\includegraphics[width=4cm]{figures/1174027/error/3_file_not_found.png}
		\centering
		\caption{FileNotFoundError}
	\end{figure}
	\item Tuliskan Kode Error dan Jenis Error
	\begin{itemize}
		\item FileNotFoundError
	\end{itemize}
	\item Cara Penangan Error
	\begin{itemize}
		\item FileNotFoundError
		\hfill\break
		Error terdapat pada kesalahan baca file csv, yang tidak terbaca. Dikarenakan letak file yang dibaca tidak para direktori yang sama. Seharusnya letakkan file di direktori yang sama. 
	\end{itemize}
\end{enumerate}
\subsection{Bukti Tidak Plagiat}
\begin{figure}[H]
	\includegraphics[width=4cm]{figures/1174027/bukti/3.png}
	\centering
	\caption{FileNotFoundError}
\end{figure}