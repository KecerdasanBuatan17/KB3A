\section{1174002 – Habib Abdul Rasyid}

\subsection{Teori}
\begin{enumerate}
\item Jelaskan dengan ilustrasi gambar sendiri apa perbedaan antara vanilla GAN dan cGAN\\
Vanilla GAN: Ini adalah jenis GAN yang paling sederhana. 
Di sini, Generator dan Diskriminator adalah perceptron multi-layer sederhana. 
Dalam vanilla GAN, algoritma ini sangat sederhana, ia mencoba untuk mengoptimalkan persamaan matematika. 
Conditional GAN (CGAN): CGAN dapat digambarkan sebagai metode pembelajaran yang mendalam di mana beberapa parameter 
bersyarat ditempatkan. Di CGAN, parameter tambahan ‘y’ ditambahkan ke Generator untuk menghasilkan data yang sesuai. 
Label juga dimasukkan ke dalam input ke Diskriminator agar Diskriminator membantu membedakan data nyata dari data yang 
dihasilkan palsu.

\begin{figure}[H]
\includegraphics[width=4cm]{figures/1174002/9/1.png}
\centering
\caption{Vanilla GAN dan cGAN}
\end{figure}

\item Jelaskan dengan ilustrasi gambar sendiri arsitektur dari Age-cGAN\\
Age-cGANs juga dapat digunakan untuk membangun sistem Face Aging, 
sintesis usia dan perkembangan usia memiliki banyak aplikasi industri dan konsumen yang praktis.
seperti pengenalan wajah lintas usia, menemukan anak yang hilang, hiburan, efek visual dalam film.
untuk lebih jelasnya bisa dilihat pada ilustrasi gambar dibawah ini.

\begin{figure}[H]
\includegraphics[width=4cm]{figures/1174002/9/2.png}
\centering
\caption{Arsitektur Age-cGAN}
\end{figure}

\item Jelaskan dengan ilustrasi gambar sendiri arsitektur encoder network dari Age-cGAN\\
Encoder: Ini mempelajari pemetaan terbalik dari gambar wajah input dan kondisi usia dengan vektor laten Z.
Jaringan encoder menghasilkan vektor laten dari gambar input. Jaringan Encoder adalah CNN yang mengambil gambar dari dimensi (64, 64, 3) dan mengubahnya menjadi vektor 100 dimensi.
Ada empat blok konvolusional dan dua lapisan padat.
Setiap blok konvolusional memiliki lapisan konvolusional, diikuti oleh lapisan normalisasi batch, dan fungsi aktivasi kecuali lapisan konvolusional pertama.

\begin{figure}[H]
\includegraphics[width=4cm]{figures/1174002/9/3.png}
\centering
\caption{Arsitektur Encoder Network dari Age-cGAN}
\end{figure}

\item Jelaskan dengan ilustrasi gambar sendiri arsitektur generator network dari Age-cGAN\\
Pada generator dibutuhkan representasi tersembunyi dari gambar wajah dan vektor kondisi sebagai input dan menghasilkan gambar. generator adalah CNN dan dibutuhkan vektor laten 100 dimensi dan vektor kondisi y, dan mencoba menghasilkan gambar realistis dari dimensi (64,64,3). generator memiliki lapisan padat, membingungka dan konvlolusional. lalu dibutuhkan dua input satu adalah vektor noise dan yang kedua adalah vektor kondisi. vektor kondisi adalah informasi tambahan yang disediakan untuk jaringan. untuk Age-cGAN ini akan menjadi age.

\begin{figure}[H]
\includegraphics[width=4cm]{figures/1174002/9/3.png}
\centering
\caption{Arsitektur Generator Network dari Age-cGAN}
\end{figure}

\item Jelaskan dengan ilustrasi gambar arsitektur discriminator network dari
Age-cGAN\\
Tujuan utama dari jaringan diskriminator adalah untuk mengidentifikasi apakah gambar yang disediakan adalah palsu atau nyata. Hal ini dilakukan dengan melewati gambar melalui serangkaian lapisan sampling bawah dan beberapa lapisan klasifikasi. Dengan kata lain, ini memprediksi Apakah gambar itu nyata atau palsu. Seperti jaringan lain, Jaringan diskriminator lain dalam jaringan convolutional. Ini berisi beberapa blok convolutional. Setiap blok convolutional berisi lapisan convolutional, lapisan normalisasi batch, dan fungsi aktivasi, selain blok convolutional pertama, yang tidak memiliki lapisan normalisasi batch. 


\item Jelaskan dengan ilustrasi gambar sendiri apa itu apa itu pretrained Inception-ResNet-2 Model\\
pre-trained Inception-ResNet-2 network, sekali disediakan dengan gambar, mengembalikan yang sesuai embedding. Tertanam yang diekstrak untuk gambar asli dan gambar direkonstruksi dapat dihitung dengan menghitung jarak Euclidean dari yang tertanam.

\begin{figure}[H]
\includegraphics[width=4cm]{figures/1174002/9/5.png}
\centering
\caption{Pretrained Inception-ResNet-2 Model}
\end{figure}

\item Jelaskan dengan ilustrasi gambar arsitektur Face recognition network
Age-cGAN\\
FaceNet merupakan suatu jaringan pengenalan wajah yang mempelajari perbedaan antara gambar input x dan gambar yang direkonstruksi x. FaceNet ini dapat mengenali identitas seseorang dalam gambar yang diberikan. Model Inception, ResNet 50 atau Inception-ResNet-2 yang telah dilatih sebelumnya tanpa lapisan yang terhubung spenuhnya dapat digunakan..

\begin{figure}[H]
\includegraphics[width=4cm]{figures/1174002/9/6.png}
\centering
\caption{Arsitektur Face Recognition Network}
\end{figure}

\item Sebutkan dan jelaskan serta di sertai contoh-contoh tahapan dari Age-cGAN\\
Tahapan dari Age-cGAN adalah
\begin{itemize}
\item Input adalah semua data dan perintah yang dimasukkan yang kemudian nantinya akan diproses
\item Training, adalah suatu proses yang dimana data-data akan digunakan dalam proses training atau learning
\item Testing, adalah suatu proses yang melakukan evaluasi terhadap performa algoritma tersebut.
\end{itemize}

\item Berikan contoh perhitungan fungsi training objektif\\
Pada training network cGAn melibatkan fungsi optimalisasi. Melatih cGAN dapat dianggap sebagai permainan minimax, dimana generator dan diskriminator dilatih secara bersamaan. Dalam persamaan dibawah ini, a merupakan parameter dari jaringan generator, dan n mewakili parameter G dan D, logD(r) adalah kehilangan dalam model generator dan Pdata adalah distribusi dari semua gambar yang mungkin.

\begin{figure}[H]
\includegraphics[width=4cm]{figures/1174002/9/7.png}
\centering
\caption{Perhitungan Fungsi Training Objektif}
\end{figure}

\item Berikan contoh dengan ilustrasi penjelasan dari Initial latent vector approximation\\
Initial latent vector approximation adalah suatu metode untuk memperkirakan vektor laten untuk mengoptimalkan rekonstruksi gambar wajah. untuk memperkirakan vektor latent, kami memiliki jaringan pembuat encode. yaitu dengan melatih jaringan encoder pada gambar yang dihasilkan dan gambar nyata. setelah dilatih, jaringan encoder akan menghaslkan vektor laten dari distribusi bersandar. fungsi tujuan training untuk melatih jaringan encoder yaitu kehilangan jarak euclidean.

\item Berikan contoh perhitungan latent vector optimization\\
Selama optimasi vektor laten, dengan mengoptimalkan jaringan encoder dan jaringan generator secara bersamaan. persamaan yang kami gunakan untuk optimasi vektor laten adalah sebagai berikut :

\begin{figure}[H]
\includegraphics[width=4cm]{figures/1174002/9/8.png}
\centering
\caption{Perhitungan latent vector optimization}
\end{figure}
Pada persamaan diatas menunjukkan bahwa jarak euclidean antara gambar asli dan gambar yang direkonstruksi harus minimal. pada tahap ini, kita bisa mencoba meminimalkan jarak untuk memaksimalkan pelestarian identitas.

\end{enumerate}


\subsection{Praktek}
\begin{enumerate}
\item Nomor 1\\
\hfill\break
\lstinputlisting[firstline=25, lastline=30]{src/1174002/9/1174002.py}
Pada kode diatas yaitu menghubungkan google drive dan mengextract dataset. adapun langkah-langkahnya bisa dilihat pada gambar berikut :
\begin{itemize}
\item Pertama, login terlebih dahulu ke akun google masing-masing dan masuk ke google colab
\item sambungkan google drive dengan google colab
\item Melakukan proses extract melalui notebook python di google colab. untuk mengestract bisa menggunakan codingan seperti pada kode diatas
\end{itemize}

\item Nomor 2\\
\hfill\break
\lstinputlisting[firstline=33, lastline=57]{src/1174002/9/1174002.py}
Maksud dari kode diatas yaitu untuk melakukan load data dan melakukan fungsi perhitungan usia

\item Nomor 3\\
\hfill\break
\lstinputlisting[firstline=60, lastline=99]{src/1174002/9/1174002.py}
Maksud encoder dalam kode diatas yaitu untuk mempelajari pemetaan terbalik dari gambar wajah yang diinput dan kondisi usia dengan vektor laten Z

\item Nomor 4\\
\hfill\break
\lstinputlisting[firstline=102, lastline=140]{src/1174002/9/1174002.py}
Maksud generator dalam kode diatas yaitu generator network mampu bekerja dengan baik dengan membutuhkan representasi tersembunyi dari gambar wajah dan vektor kondisi sebagai input dan menghasilkan gambar

\item Nomor 5\\
\hfill\break
\lstinputlisting[firstline=143, lastline=174]{src/1174002/9/1174002.py}
Maksud diskriminator pada kode diatas yaitu untuk membedakan antara gambar yang asli dan gambar yang palsu

\item Nomor 6\\
\hfill\break
\lstinputlisting[firstline=177, lastline=309]{src/1174002/9/1174002.py}
Maksud dari kode diatas yaitu sebagai proses training dengan meload file.mat pada dataset, lalu kita melakukan epoch sebanyak 500 kali.

\item Nomor 7\\
\hfill\break
\lstinputlisting[firstline=312, lastline=355]{src/1174002/9/1174002.py}
Maksud dari kode diatas yaitu dengan membuat model .h5 lalu meload data dengan menghasilkan result.

\end{enumerate}

\subsection{Penanganan Error}
eror terjadi karena kurang teliti, maka sangat disarankan untuk teliti dan tidak mudah menyerah, okey :D

\subsection{Bukti Tidak Plagiat}
\begin{figure}[H]
\includegraphics[width=4cm]{figures/1174002/9/plagiarisme.png}
\centering
\caption{Bukti Plagiarisme}
\end{figure}

\subsection{Link Youtube}

